%%%%%%%%%%%%%%%%%%%%%%%%%%%%%%%%%%%%%%%%%%%%%%%%%%%%%%%%%%%%%%%%%%%%%%%
%%%%%%%%% Aşağıda istenilen bilgileri dikkatlice doldurunuz.   %%%%%%%%
%%%%%%%%% Doldurmanız istenilen ifadenin sonunda TR ya da EN   %%%%%%%%
%%%%%%%%% yazıyorsa, sırasıyla Türkçe veya İngilizce olarak    %%%%%%%%
%%%%%%%%% doldurunuz. Eğer herhangi bir ifade yoksa, tezinizi  %%%%%%%%
%%%%%%%%% hangi dilde yazıyorsanız (Türkçe veya İngilizce), o  %%%%%%%%
%%%%%%%%% dile göre doldurunuz. İsimleri yazarken soyisimleri  %%%%%%%%
%%%%%%%%% büyük harf ile yazınız.                              %%%%%%%%
%%%%%%%%%%%%%%%%%%%%%%%%%%%%%%%%%%%%%%%%%%%%%%%%%%%%%%%%%%%%%%%%%%%%%%%
%%%%%%%%%%%%%%%%%%%%%%%%%%%%%%%%%%%%%%%%%%%%%%%%%%%%%%%%%%%%%%%%%%%%%%%


% Tez başlığını Türkçe olarak yazınız.
\def\titleTR{Otonom Depo Robotları için Simülasyon, Yer Belirleme ve Haritalama}

% Tez başlığını İngilizce olarak yazınız.
\def\titleEN{Simulation, Localization and Mapping for Automated Warehouse Robots}

% İsminizi yazınız.
\def\student{Ramin MAMMADZADA}
% Anabilim dalınızın İngilizce adını yazınız.
\def\departmentEN{Department of Mechatronic Engineering}
% Anabilim dalınızın Türkçe adını yazınız.
\def\departmentTR{Mekatronik Mühendisliği Anabilim Dalı}
% Programınızın İngilizce adını yazınız.
\def\programEN{Program of Mechatronic Engineering }
% Programınızın Türkçe adını yazınız.
\def\programTR{Mekatronik Mühendisliği Programı }
% Tez sınavı tarihini yazınız. (gg.aa.yyyy)
\def\dateFull{10.05.2019}
% Tez sınavı tarihini, tez için kullandığınız dilde şu formatta yazınız. (ay, yıl)
\def\date{May, 2019}

% Tezinizde eş-danışman varsa 1, yoksa 0 yazınız.
\def\isThereCoAdvisor{0}

% Tez danışmanınızın ismini Türkçe ünvanı ile yazınız.
\def\advisorTR{Doç. Dr. Aydin Yeşildirek}
% Tez danışmanınızın ismini İngilizce ünvanı ile yazınız.
\def\advisorEN{Assoc. Prof. Dr. Aydın Yeşildirek}
% Tez danışmanınızın bağlı olduğu kurumu tez için kullandığınız dilde yazınız.
\def\advisorUni{Yildiz Technical University}

% Eş-danışmanınız varsa ismini Türkçe ünvanı ile yazınız. Yoksa bu kısmı atlayınız.
\def\coadvisorTR{Doç. Dr. Eusebia MOSELEY}
% Eş-danışmanınız varsa ismini İngilizce ünvanı ile yazınız. Yoksa bu kısmı atlayınız.
\def\coadvisorEN{Assoc. Prof. Dr. Eusebia MOSELEY}
% Eş-danışmanınızın bağlı olduğu kurumu tez için kullandığınız dilde yazınız.
\def\coadvisorUni{Yildiz Technical University}

% Aşağıdaki kısma sırası ile tez sınavı üyelerinin isimlerini ünvanları ile 
% birlikte yazınız. Ardınan kişilerin bağlı bulundukları kurumları tez için 
% kullandığınız dilde yazınız. Yüksek lisans için ilk 2, doktora için ilk 4 
% bilgiyi doldurmanız gereklidir.

\def\memberi{Prof. Dr. Haydar Livatyalı}
\def\memberiUni{Yildiz Technical University}

\def\memberii{Prof. Dr. Vasfi Emre Ömürlü}
\def\memberiiUni{Yildiz Technical University}

\def\acknowledgementText{
    % Buraya teşekkür metninizi tez için kullandığınız dilde yazınız. 

First of all, I would like to thank to my supervisor Assoc. Prof. Dr. Aydın Yeşildirek for his endless patience, support and encouragement during this research. This thesis would not be possible without his comprehensive support.
I also would like to thank to my thesis committee members Prof. Dr. Haydar Livatyalı for his presence in our meetings and indispensable comments about this work and Prof. Dr. Vasfi Emre Ömürlü for his influence and discussions on my career decisions.
}

\def\abstractTextEnglish{
    % Buraya İngilizce olarak tez özetini yazınız.
The Simultaneous Localization and Mapping (SLAM) problem is one of the most challenging problems in robot navigation. The problem addresses autonomously exploring and mapping an unknown environment without prior knowledge. The robot should generate the map of the environment and estimate its pose with respect to the map. 

In this thesis, I will introduces the application of several successful SLAM techniques to the warehouse robots using visual sensors.

}

\def\abstractKeywordsEnglish{
    % Buraya İngilizce olarak tez için geçerli anahtar kelimeleri yazınız
    SLAM, mapping, warehouse automation, mobile robots
}

\def\abstractTextTurkish{
    % Buraya Türkçe olarak tez özetini yazınız
    Gerçek zamanlı yer bulma ve haritalama (SLAM) problemi robot gezinmesi alanındaki en zor problemlerden biridir. Bu problem önceden ortamla ilgili her hangi bir bilgiye sahip olmadan, ortamın otomatik olarak taranıp, incelenmesi, haritasının çıkarılması ve robotun yer bilgisinin bulunmasını içerir. Bu tezde, bir çok SLAM yöntemlerinin, farklı algılayıcılarla donatılmış depo robotları üzerinde uygulamasını bulabilirsiniz.

}

\def\abstractKeywordsTurkish{
    % Buraya Türkçe olarak tez için geçerli anahtar kelimeleri yazınız
    SLAM, haritalama, depo otomasyonu, mobil robotlar
}

% Tez için kullandığınız dilde, aşağıdaki kısma tez için aldığınız destekleri yazınız. Eğer destek almadıysanız küme parantezleri içerisindeki yazıları siliniz.
\def\supports{}

% Tez için kullandığınız dilde, tezin ithaf metnini yazınız. Satır atlatmak için \\ kullanabilirsiniz.
\def\dedicationText{Dedicated to my family}

%%%%%%%%%%%%%%%%%%%%%%%%%%%%%%%%%%%%%%%%%%%%%%%%%%%%%%%%%%%%%%
%%%% Aşağıdaki alana "\item[sembol] Sembol_açıklaması" %%%%%%%
%%%% şeklinde sembollerinizi giriniz. Açıklamanın ilk  %%%%%%%
%%%% harfine göre sıralayınız. Eğer sembol kullanmı-   %%%%%%%
%%%% yorsanız "\def\symbols{}" olacak şekilde küme     %%%%%%%
%%%% parantezlerinin içini siliniz.                    %%%%%%%
%%%%%%%%%%%%%%%%%%%%%%%%%%%%%%%%%%%%%%%%%%%%%%%%%%%%%%%%%%%%%%

\def\symbols{

    \begin{abbrv}
        
        \item[$t$]                Time step at the execution of a discrete time system
        \item[$x_{t}$]          State vector at time t
        \item[$z_{t}$]          Observation vector at time t
        \item[$u_{t}$]          Control input at time t.
        \item[$bel(x_{t})$]     Belief state at time t
        \item[$\mu_{t}$]        Mean vector of the belief state at time t
        \item[$\epsilon_{t}$]   Process noise vector at time t
        \item[$\delta_{t}$]     Observation noise vector at time t
        \item[$Q_{t}$]          Observation noise covariance at time t
        \item[$g(x_{t-1},\mu_{t})$]        Non-linear process function.
        \item[$h(x_{t})$]       Non-linear observation function.
        \item[$G_{t}$]          Jacobian of g() with respect to xt at time t.
        Particle Filter. 
        \item[$Z_{t}$]          Observation set. Observation sigma point set in UKF. 
        \item[$K$]              Number of observations.
        \item[$w_{i}$]          Importance weight of particle i.
        \item[$N$]              Number of particles.
        \item[$p$]          Robot pose.
        
        \item[$p_{x}$] x coordinate of the robot pose.
        \item[$p_{y}$] y coordinate of the robot pose.
        \item[$p_{\theta}$] Orientation of the robot.
        \item[$\theta_{i}$] Orientation of the ith landmark with respect to the robot. 
        \item[$l_{i}$] Distance of the ith landmark to the robot.
        \item[$\Delta x$] Displacement of a robot in the x axis.
        \item[$\Delta y$] Displacement of a robot in the y axis.
        \item[$\Delta \theta$] Change in the orientation of a robot. 
        \item[$M$] Map.
        \item[$m_{i}$] ith entity in the map.
        \item[$\ p_x^i\ $] x coordinate of the ith landmark.
        \item[$\ p_y^i\ $] y coordinate of the ith landmark.
                
    \end{abbrv}

}


%%%%%%%%%%%%%%%%%%%%%%%%%%%%%%%%%%%%%%%%%%%%%%%%%%%%%%%%%%%%%%
%%%% Aşağıdaki alana "\item[kısaltma] kısaltma_açıklaması" %%%
%%%% şeklinde kısaltmalarınızı giriniz. Kısaltmanın ilk    %%%
%%%% harfine göre sıralayınız. Eğer kısaltma kullanmıyor-  %%%
%%%% sanız "\def\abbrevations{}" olacak şekilde küme pa-   %%%
%%%% rantezlerinin içini siliniz.                          %%%
%%%%%%%%%%%%%%%%%%%%%%%%%%%%%%%%%%%%%%%%%%%%%%%%%%%%%%%%%%%%%%
\def\abbrevations{

    \begin{abbrv}
        \item[KF] Kalman Filter
        \item[EKF] Extended Kalman Filter
        \item[UKF] Unscented Kalman Filter
        \item[SLAM] Simultaneous Localization and Mapping
    \end{abbrv}
}