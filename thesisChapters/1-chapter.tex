\chapter{Introduction}

One of the most challenging problems in robot navigation is the simultaneous localization and mapping problem (SLAM). The SLAM problem attracted many researchers in the field of mobile robotics. The problem addresses a robot generating a map of an unknown environment and localizing itself in this map.
In the localization problem, the robot is given a sequence of possibly noisy landmark observations, and for all the observations, the location of the landmarks are known more or less prior to execution. However, if the landmark positions are not available, the robot should both estimate the real positions of the landmarks and localize itself with the estimated landmarks.

From the other point of view, in the mapping problem, the position of the robot is given, and it estimates the positions of the unknown landmarks. If the robot position and orientation (pose) is not given, the robot needs to estimate its pose using the landmarks. Since the odometer readings are noisy in real applications, the error in the estimations are unbounded. The problem attracts researchers because the solution provides more autonomy to robots and more realistic applications.

The SLAM problem addresses the estimation of both unknown landmarks in the environment and the robot pose with noisy observations and odometer readings. Depending on the application domain, the environment features are unknown and the observations are indistinguishable, meaning that the landmarks may not be unique. Most of the solution methods to this problem make probabilistic estimations with different assumptions on probability distributions of both the noise and the estimations.

The scanning range finders are suitable for mapping applications, however visual sensors are cheaper and can provide more information of the environment than range finders. However, observing a feature in an image of unknown environment requires Computer Vision algorithms which are computationally demanding methods. 

The SLAM problem may be in many forms according to the application domain or the task of the robot. The robot may be sent to explore a cave without being lost or it might be expected from the robot to generate a precise map of a warehouse. In this thesis, I will focus on the two basic SLAM solution methods are applied to a real robot platform and the experimental study is done on both simulation and real world. I will provide visual landmarks for the robot simulation and inspect the accuracy and the applicability of the methods on warehouse ground robot. In Chapter 2, I will give exact definition of SLAM problems and their solutions from previous researches which are done by other researchers. The specifications and limitations of our problem definition and the underlying hardware and software platform will be detailed in future.